\chapter*{bbbbb}
\addcontentsline{toc}{chapter}{\protect\numberline{}Introduction}

La communication sans fil est un moyen de permettre à plusiuers appareils distincts d'échanger des données sans avoir à les relier
entre eux. Elle utilise généralement des micro-ondes mêmes si il est possible de faire passer une information par d'autres bandes de
fréquence comme les infra rouges. Le wifi, pour wireless fidelity, est un protocole de la couche d'accés de l'architecture TCP/IP
couramment utilisé permettant à plusieurs appareils de communiquer via les micro-ondes. Il dispode de deux modes de fonctionnement : 
Le mode centralisé, et le mode ad-hoc, que l'on appelle aussi mode mesh, ou maillé. Le premier est trés majoritairement utilisé. 
Cependant, il posséde des limites, nottement la nécessité de disposer d'un point d'accés, rendant interssant l'utilisation du mode
ad-hoc dans certaines conditions. En effet, avec le développement rapide des objets connectés, seul ce dernier permet à deux objets
suffisament proches de communiquer sans avoir besoin d'appareils dédié à relayer les trames.

Le mode mesh est standardisé par l'IEEE avec la norme IEEE 802.11s. Le but de ce projet est de créer un framework permettant de
configurer simplement et automatiquement un réseau wifi mesh sur un systéme d'exploitation linux. Plus précisément, il devra proposer
des fonctions pour détecter les cartes réseaux sans fils dont dispose la machine, détecter les réseaux wifi mesh déja existant,
rejoindre un réseau mesh déja existant ou en créer un nouveau sur le canal le moins encombré, associer une adresse IP aux interfaces
sans fils et trouver les chemins permettant d'acheminer les trames jusqu'à leur destination.



