\chapter{Objectifs, definitions, contraintes}
\section{Introduction aux réseaux wifi}

Le wifi, abréviation de wireless fidelity, est un ensemble de protocoles permettant la communication sans fil entre deux
appareils en utilisant des ondes radios. Ces protocoles se situent au niveau de la couche d'accés du modéle tcp/ip.
La standardisation de cette norme à été initié l'IEEE\footnote{Institute of Electrical and Electronics Engineers} en 1990.
Cela à aboutit, en 1997, au standart IEEE 802.11 définissant les réseaux locaux sans fils \cite{WFintro}.
La norme d'origine prévoyait l'utilisation d'ondes radios dans la bande de fréquences libre entre 2401 et 2495 MHz\cite{WFband},
courrament appelée bande à 2,4 GHz, ou d'infra rouges. Cependant, pour suivre l'évolution des technologies, le standart IEEE 
802.11 s'est enrichi afin d'augmenter le débit et d'utiliser la bande de fréqences libre entre 5170 et 5710 MHz.
Les standarts IEEE 802.11a et IEEE 802.11b à donc été définit en 1999, le standart 802.11g et 2003 et le standart 802.11n
en 2009.

Depuis sa création, la norme IEEE 802.11 définit 14 cannaux dans la bande 2,4 GHz. Chaque canal à une largeur de 22 MHz et 
l'écart entre les centres de deux canaux successifs est de 5 MHz\footnote{Sauf les centres des cannaux 13 et 14 qui sont espacé
de 12 MHz}. Il en résulte donc un fort recouvrement entre les différents cannaux comme le montre la figure 1.1.

\begin{figure}
   \centering
   \includegraphics[width=0.8\textwidth,natwidth=610,natheight=642]{images/cannaux.png}
   \caption{Répartition des cannaux dans la bande 2,4 GHz}
\end{figure}

Un réseau wifi est un réseau local découpé en ``cellules'' appelée BSS\footnote{Basic Service Set}. Deux appareils
doivent se trouver dans le même BSS pour communiquer entre eux. Il existe deux modes de BSS : Le mode Infrastructure et le
mode ad-hoc\cite{WFfunc2}. La pluspart des réseaux wifi de particuliers ou d'entreprises sont des réseaux en mode Infrastructure.

Le mode infrastructure est une topologie centralisé. Il se caractérise par le fait que chaque BSS posséde une station de
base, appelé aussi point d'accés, et que toutes les communications passent nécessairement par le point d'accés de la BSS,
et ce même si l'émeteur et le récepteur du message se trouvent dans le même BSS. Un point d'accés peut être relié par un réseau
cablé à un ou plusieurs autres points d'accés, étendant ainsi le LAN\footnote{Local Area Network, ou Réseau local}
\cite{WFfunc},ou à un routeur pour accéder à un réseau WAN\footnote{Wide Area Network, ou Réseau étendu}.
Le mode ad-hoc, au contraire, est un mode ``d'égal à égal''. Deux entités au sein du même BSS peuvent communiquer directement.


Comme le montre la figure 1.2\cite{WFhead}, le premier champ de l'en tête wifi est le FCF\footnote{Frame Control Field, ou Champ
de contrôle de trame}, permettant d'identifier les trames en fonction de leur rôle. Ainsi, les trames peuvent être de trois types,
identifiées par les deux bits en position 3 et 4 du FCF : Management, Contrôle ou Donées.\cite{WFfcf}. Les 4 bits suivants
identifient le sous type, et les 8 derniers bits sont des flags. Les trames de données sont utilisé pour transporter des données
de plus haut niveau. Les trames de contrôles sont utiliser pour les acquittements et les réservations, et les trames de management
servent à organiser et maintenir le réseau\cite{MNfunc}.

Les Beacons frames sont des trames de management particuliéres qui permettent à un point d'accés de déclarer sa présence aux
appareils à proximité. Ils transportent différentes informations comme le SSID\footnote{Service Set IDentifier} du réseau,
qui est une chaîne de 2 à 32 charactéres, un timestamp permettant de se synchroniser, le canal sur lequel il émet, 
et d'autres informations.\cite{WFfunc2}.
\begin{figure}
   \centering
   \includegraphics[width=0.8\textwidth,natwidth=488,natheight=513]{images/header_wifi.png}
   \caption{Format des trames 802.11}
\end{figure}

\section{La norme 802.11s}
Comme dit précédamment, le mode infrastructure est actuellement le plus utilisé. Cependant, il posséde des limites du au fait
que, dans certaines situations, il n'est pas toujours possible de connecter un point d'accés à un switch\cite{MNintro}.
En effet, la longueur des cables ethernet est limité, ce qui rend difficile le déploiement de points d'accés dans des
environnement ouverts.

C'est ce qui fait la force du mode ad-hoc. Chaque appareil peut communiquer avec tous les autres appareils qui sont à portée.
De plus, chaque appareil peut relayer le message si le destinataire final n'est pas à portée. Ainsi, si l'on prends l'exemple
la figure 1.3, chaque noeud peut communiquer avec nimporte quel autre, à la condition qu'un alogrithme de routage s'execute
sur le réseau et que chaque noeud sache quel est le suivant pour atteindre la destination. Ce genre de réseau est appelé réseau
maillé\footnote{Mesh Network en anglais}. Le gros avantage de ces réseaux est qu'ils sont trés flexible. On peut les étendre sans
avoir à tirer de nouveaux cables ou a ajouter de nouveaux equipements intermédiaires\cite{MNintro}. A l'inverse, leretrais d'un
petit nombre de noeuds ne doit pas empécher le réseau de fonctionner si il est possible de trouver des routes alternatives pour
les trames.
\begin{figure}
   \centering
   \includegraphics[width=0.8\textwidth,natwidth=488,natheight=513]{images/ad_hoc.png}
   \caption{Exemple de réseau ad-hoc}
\end{figure}

Le standart 802.11s est un amendement de la norme 802.11, définissant la maniére dont les appareils disposant d'une carte réseau
sans fils peuvent s'interconnecter pour former un réseau sans fil maillé. L'IEEE a commencé a travailler sur ce standart en 
2003 et celui-ci à été adopté en 2006. Pour faciliter l'interopérabilité, un réseau 802.11s est vu de l'extérieur comme un
unique segment ethernet. Pour permettre la retransmission des informations d'un noeud à l'autre, la norme 802.11s
étends l'en tête 802.11 classique avec un en tête mesh comme montré dans la figure 1.4\cite{MNfunc}.

Les 4 champs d'adresses de l'en tête 802.11 sont utilisées, puisqu'il faut à chaque transmission du message donner l'adresse
du noeud qui à effectué la transmission, du prochain noeud, du destinataire final et de l'expéditeur originel. Dans certains cas
plus complexes, par exemple si l'émeteur ou le destinataire, ou les deux, ne se trouvent pas dans le réseau mesh, mais que la 
trame va traverser un réseau mesh, il faut ajouter des adresses supplémentaires, d'ou le fait que l'en tête mesh comporte un 
champ optionel d'extention d'adresse. Parmis les autres valeurs ajoutées, le TTL \footnote{Nombre de fois maximal que peut être
relayé une trame avant d'être abandonnée, cette valeur est décrémenté à chaque saut} et le Mesh sequence number\footnote{nombre
identifiant de maniére unique un paquet} permettent d'éviter les boucles infinies qui risqueraient de saturer le réseau.


\begin{figure}
   \centering
   \includegraphics[width=0.8\textwidth,natwidth=488,natheight=513]{images/mesh_header.jpeg}
   \caption{Format des trames 802.11s}
\end{figure}

\section{Adressage et routage}

Dans un réseau TCP/IP, chaque noeud doit disposer de deux adresses. Chacune permet de l'identifier, en théorie, de maniére 
unique. La premiére est l'adresse MAC, une adresse sur 48 bits, utilisé pour identifier les noeuds dans les protocoles de la 
couche d'accés du modéle TCP/IP. Cette adresse permet à une trame de voyager sur un LAN jusqu'à sa destination, mais sera changé
à chaque fois que le paquet passe par un routeur. La deuxiéme est l'adresse IP, une adresse sur 32 bits qui est utlisé par le
protocole IP, qui est un protocole de la couche réseau du modéle TCP/IP. Cette adresse est inchangé d'un bout de la transmission
à l'autre\footnote{en l'absence de mécanismes de traduction d'adresse (NAT)}.

L'adresse MAC est atribué à une carte réseau par le constructeur. Ainsi, nous avons la garantie que chaque appareil posséde une
adresse MAC unique. L'adresse IP doit également être unique mais, contrairement à l'adresse MAC, elle n'est pas enregistré dans
la carte réseau par le constructeur car toutes les adresse IP identifiant les appareils d'un même LAN doivent avoir le même 
préfixe. Il existe des protocoles permettant d'affecter automatiquement des adresses IP à des appareils sans avoir besoin de
recourir à une intervention humaine. Le protocole majoritairement utilisé est DHCP\footnote{Dynamic Host Configuration 
Protocol}. Ce protocole nécessite qu'un serveur dispose d'une liste d'adresses IP disponible qu'il va affecter à chaque noeuds
du réseau sur demande de ces derniers\cite{DHCP}. Néamoins, le recours à un serveur central d'adresse IP amoindrit les 
avantages à l'utilisation d'une infrastructure décentralisé tel qu'un un réseau maillé.

Dans un réseau maillée, il est aussi nécessaire de prévoire le routage des trames. La norme 802.11s définit également le protocole
HWMP\footnote{Hybrid Wireless Mesh Protocol} comme protocole de routage pour les réseaux wifi maillées. Contrairement à la majorité 
des protocoles de routages, HWMP ne se base pas sur les adresses IP, mais sur les adresses MAC, puisque le but est d'aiguiller
les trames au sein d'un même LAN. Il s'agit d'un protocole de routage à vecteur distance puisque les noeuds n'ont pas conaissance
de l'intégralité de la topologie du réseau mais uniquement des noueds qui le constituent et de la ``distance'' de chacun d'eux
\cite{MNroute}.
