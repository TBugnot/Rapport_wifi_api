\chapter{Réseaux sans fils sous Linux}
\section{Gestion du réseau sous Linux}

Une grosse partie du projet consiste à pouvoir communiquer avec les interfaces sans fils dont dispose les appareils. En effet,
il est nécessaire d'une part, de récupérer des informations à propos de ces derniéres et d'autre part, de les configurer pour
arriver à les utiliser de la maniére que nous le souhaitons. Historiquement, sous linux, la communication avec les interfaces
se faisait avec des appels systémes ioctl\footnote{Abréviation de Input Output ConTroL, il s'agit d'une fonction permettant 
de manipuler des fichiers spéciaux}. Des outils permettant de manipuler les interfaces en utilisant cette méthode sont depuis 
longtemps fournis avec les distribution linux. C'est le cas par exemple du package net-tools, incluant le programme ifconfig,
et permettant de manipuler les interfaces (état, informations d'adressages) ou de wireless-tools, incluant le programme iwconfig,
permettant de mainpuler plus précisément les interfaces sans fils.

Cependant, depuis 2007, il existe un autre moyen de manipuler les interfaces. En effet, se développe \textbf{netlink}, une
famille de socket ayant pour but de fare communiquer les processus entre eux. Cela permet, entre autre, de faire communiquer
un processus utilisateur avec un processus du noyeau linux. La librairie libnl implémente les pré-requis fondamentaux pour 
utiliser le protocole netlink. Cependant, celle ci se veux minimaliste. C'est pourquoi elle est complété par 3 API : libnl-route,
libnl-genl et libnl-nf\cite{NLlibs}. Des outils de configuration d'interfaces utilisent ces nouveaux moyens. C'est la cas de 
la suite d'outil iproute pour controles les interfaces et de iw pour controler plus précisément les interfaces sans fils. Nous 
utiliserons netlink dans notre code.

L'API que nous utiliserons principalement pour gérer les interfaces est libnl-genl. Celle ci permet, grace à nl80211, un
en-tête netlink, d'envoyer des messages permettant de contrôler les interfaces wifi. 

\section{Création de réseaux 802.11s}
\section{Détection de réseaux existants et sélection de canal}