\chapter{Adressage et routage}
\section{Adressage dans un réseau maillé}

Dans la partie 3 du chapitre 1, nous avons évoqué l'utilité de donner aux interfaces des adresses IP. Sans elles, il est
impossible de leur faire transmettre des paquets de données. Nous évoquions également le protocole DHCP, qui est le protocole le plus
couramment utilisé pour affecter dynamiquement des adresses IP aux machines qui constituent le réseau. Puisque celui-ci nécessite un
serveur dédié, il n'est pas adapté à une architecture décentralisée comme un réseau wifi mesh.

Cependant, il existe de nombreuses solutions plus adaptées à nos besoins. Ces solutions sont regroupées sous l'appelation zeroconf
\footnote{Ou Zero-configuration networking}. Elles permettent de créer un réseau IP sans avoir recours à un serveur.
Les principales fonctionnalités de ces solutions sont l'affectation d'adresses IP sans serveur DHCP et la résolution de nom de 
domaine sans serveur DNS. Seule la premiére nous intéresse dans le cadre de ce projet.

La solution qui a été retenue  est ``IPv4 link-local autoconfiguration specification''\footnote{Abrégé en
IPv4LL} définie par l'IETF\cite{ipv4ll}. IPv4LL permet aux clients de choisir une adresse IP dans la plage d'adresse 169.254.0.0/16.
Cette plage est réservée auprès de l'IANA\footnote{Internet Assigned Numbers Authority} pour l'adressage local. Le principe de IPv4LL
est de choisir une adresse IP aléatoirement dans la plage réservée, puis d'envoyer une requette ARP sur cette même adresse pour savoir
si elle est déja utilisée par un autre noeud. En l'absence de réponse, l'adresse IP peut être utilisée. Sinon, il faut en choisir une
autre et envoyer à nouveau une requette ARP.

Avahi est un programme fourni avec la plupart des distributions linux permettant, entre autre, d'affecter des adresses IP aux
interfaces en utilisant IPv4LL. Il peut être lancé sous la forme d'un ``deamon''\footnote{Un processus ou ensemble de processus
s'executant en arrière plan}, en lui précisant l'interface à laquelle il doit donner une adresse IP. Durant son execution, il
affecte une adresse ip à l'interface en vérifiant que celle-ci est libre, puis répond aux requettes si une autre instance de avahi
tournant sur une autre machine demande à utiliser la même adresse. Nous avons donc choisi, dans le cadre de ce projet, de lancer
avahi, via un appel à la fonction system, afin d'affecter des adresses IP aux interfaces. Au besoin, l'option -k permet d'arrêter
les instances d'avahi tournant sur une interface.

\section{Routage}

Dans le chapitre 1, partie 3, nous précisions aussi qu'il faut permettre de relayer les trames dans un réseau maillé. La norme
802.11s définit un protocole de routage, le protocole HWMP. Celui-ci est un protocole à vecteur distance, ce qui signifie que les
noeuds n'ont pas conaissance de l'intégralité de la topologie. La ``table des chemins\footnote{Puisqu'on se situe au niveau de la 
couche d'accés du modèle TCP/IP, on parle plus souvent de chemins (path) que de routes (route)}'' contient, pour chaque destination,
deux informations essentielles : Le prochain saut et la distance de la destination.

Plus précisément, le protocole HWMP dispose de deux modes de fonctionnement : ``on demand'' et ``proactive tree building''. Le
second nécessite qu'un noeud soit désigné comme noeud racine\cite{MNroute}.

Avec le mode ``on demand'', chaque fois qu'un noeud a besoin de connaître le chemin vers un autre noeud, il envoie un paquet
``route request (RREQ)'' en broadcast, en identifiant le noeud de destination. Le paquet RREQ contient aussi un champ métrique
initialisé à 0. Chaque noeud intermédiaire va recevoir le paquet RREQ, éventuellemenent en plusieurs exemplaires. Si le paquet RREQ
a une métrique plus faible que celle déjà connue, le noeud intermédiaire met à jour sa table de routage et le retransmet aprés avoir 
augmenté la métrique. Lorsque le paquet atteint le noeud de destination, ce dernier répond avec un paquet ``Route Reply (RREP)''
en unicast vers la source. Ainsi, tous les noeuds entre la destination et la source connaissent une route vers ces deux points.

Cet algorithme est contenu et executé par le noyeau linux. Cependant, il n'est pas le seul à exister pour trouver des chemins dans 
les réseaux maillés. D'autres algorithmes propriétaires lui ont précédé. C'est pourquoi nous avons programmé la
possibilité d'ajouter ou de supprimer des entrées dans la table des chemins, en précisant la destination et le prochain saut. De plus,
il est important de pouvoir récupérer cette table des chemins car elle contient les adresses MAC de toutes les machines connues du 
réseau. C'est donc un bon moyen d'avoir un aperçu de tous les noeuds qui le constituent.