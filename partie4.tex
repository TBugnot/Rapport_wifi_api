\chapter{Implémentation et architecture logicielle}
\section{Architecture globale}

Le but de ce projet est d'écrire en C un framework permettant de créer ou rejoindre facilement des réseaux mesh. Nous avons donc
décider de plusieurs régles facilitant l'exploitation du code.

Les fichiers .c contenant le code sont situés dans le dossier src. Pour chaque fichier .c, il existe un fichier .h de même nom dans
le dossier include. Les prototypes des fonctions sont définis dans ce fichier .h. De plus, au dessus de chaque prototype, des 
commentaires au format doxygen renseignent sur l'utilité et le fonctionnement de chaque fonction, détaillant les paramétres
à donner et la valeur retournée par celle-ci. 

Les fichiers interface.c, interface.h, network.c, network.h, mpath.c et mpath.h ont pour but de définir des structures 
nécessaires ainsi que les fonctions permettant d'effectuer automatiquement les opérations de bases sur ces structures : allocation, 
initialisation, affichage, libération mémoire et, lorsque les structures peuvent faire partie d'une liste, destruction de la liste.

Les fichiers wifi.c, wifi.h, scan2.c, scan2.h, ip.c, ip.h, route.c et route.h définissent les fonctions répondant aux objectifs que
nous nous sommes fixées. Ces fonctions utilisent les structures définies précédemment. Dans ces fichiers, la plupart des fonctions
commencent par le préfixe wifi\_\-. Celles qui n'ont pas ce préfixe ont un usage interne et ne sont pas destinées à être appelées depuis
l'extérieur. Certaines fonctions servent à modifier des paramètres du systéme, d'autres à obtenir des informations. Dans ce 
second cas, le ou les premiers attributs sont des pointeurs indiquant une zone mémoire qui sera modifée pour y écrire les informations.
Le type de retour est toujours un int, qui vaut 0 si l'exécution de la fonction s'est déroulée sans problème ou un nombre négatif
si une erreur  s'est produite . Ce nombre correspond à un code d'erreur et il est possible d'en obtenir une description avec 
la fonction wifi\_\-geterror.

\section{Contrôle des interfaces sans fils}

Pour créer un réseau wifi mesh ou scanner les réseaux wifi déjà existant, il faut pouvoir contrôler les interfaces sans fils. 
Cependant, avant de contrôler les interfaces, il est nécessaire de récupérer les informations les concernant. Cela passe par 
l'utilisation de l'API libnl-genl et l'envoi de messages netlink au noyeau.

Comme précisé precédemment, il faut différencier les cartes réseaux physiques des interfaces ``logiques''. Nous avons donc décidé de 
les représenter par deux structures différentes, respectivement wifi\_\-wiphy et wifi\_\-interface, décritent dans le fichier interface.h. 
Les messages netlink envoyés par le noyeau nous donnent de nombreuses informations, mais nous avons choisi de ne retenir que celles
qui seront utiles dans le cadre de ce projet. Ainsi, les cartes réseaux (wiphy) seront caractérisées par leur numéro, la liste des
fréquences utilisables et la liste de types d'interfaces supportés. Les interfaces seront, quant à elles, représentées par leur nom,
le numéro de la carte réseau associée, leur type, leur fréquence, la largeur du canal et leur adresse mac. On ajoute aussi à ces 
structures une variable list\_\-head permettant de les lier dans une liste chaînée. Le fichier interface.c contient toutes les fonctions 
permettant de manipuler simplement ces structures : Création, affichage, libération de mémoire...

Les fonctions permettant l'envoi de messages netlink au noyeau et l'analyse des réponses sont écrites dans le fichier wifi.c. Celles-ci
nécessitent l'utilisation d'un socket netlink. Pour éviter d'avoir à créer le socket à chaque appel d'une fonction, nous avons pris la
décision de créer une structure, que nous avons appelée wifi\_\-nlstate, contenant une référence sur le socket ainsi qu'un identifiant
qui est également nécessaire pour l'envoi de messages netlink. Cette structure est définie dans le fichier wifi.h. Nous avons aussi
créé une fonction permettant d'initialiser cette structure. Ainsi, l'utilisateur doit  appeler une seule fois cette fonction 
d'initialisation, puis fournir un pointeur sur la structure à chaque appel d'une autre fonction nécessitant d'envoyer un message
netlink.

La fonction send\_\-recv\_\-msg permet d'envoyer un message netlink et d'analyser la ou les réponses. Cette fonction prend en paramétre 
un pointeur sur la structure wifi\_\-nlstate décrite dans le paragraphe précédent, la commande nl80211 à mettre dans le message
netlink, les flags ainsi qu'une liste d'attributs à ajouter dans ce message, une réfèrence sur la fonction permettant d'analyser la 
réponse et un argument à fournir à cette fonction. Plusieurs fonctions définies dans le fichier wifi.c utilisent cette fonction
send\_\-recv\_\-msg. Le but et fonctionnement de chacune d'elle sont donnés en commentaire dans le fichier wifi.h. Elles permettent, 
entre autre, de récupérer la
liste des cartes réseaux (wifi\_\-get\_\-wiphy) et interfaces (wifi\_\-get\_\-interfaces), de limiter ces listes uniquement aux cartes réseaux 
supportant un type particulier (wifi\_\-get\_\-wiphy\_\-supporting\_\-type) ou aux interfaces utilisant ces cartes réseaux
(wifi\_\-get\_\-if\_\-supporting\_\-type). Il est également possible de changer le type (wifi\_\-change\_\-type) ou la fréquence de fonctionnement
(wifi\_\-change\_\-frequency) d'une interface, de lui associer un MeshID (wifi\_\-set\_\-meshid) ou d'en créer une nouvelle 
(wifi\_\-create\_\-interface). Seules les fonctions permettant d'allumer (wifi\_\-up\_\-interface) ou d'éteindre (wifi\_\-down\_\-interface) une 
interface utilisent un appel système à la place d'un message netlink.

\section{Scan des réseaux}

De même que pour les interfaces, nous avons du créer des structures pour représenter les réseaux. Ces structures, wifi\_\-network et 
wifi\_\-mesh\_\-network, permettant de représenter respectivement des réseaux wifi centralisés et maillés, sont définies dans le fichier
network.h. Les beacons frames contiennent plusieurs informations. Dans le cadre des réseaux centralisés, puisque nous ne cherchons
à les détecter que pour savoir quelle fréquence est la moins occupée, nous nous contenterons de les représenter par leur SSID et leur
canal. Par contre, un réseau mesh sera représenté par son MeshID, son canal, ainsi que tous les paramètres mesh fournis dans la
beacon frame. Ces deux structures contiennent aussi une variable list\_\-head leur permettant de faire partie d'une liste.
Le fichier network.c contient les fonctions permettant de créer, d'initialiser, d'afficher de supprimer ces structures.

Le fichier scan2.c définit les fonctions permettant d'effectuer un scan et de lister les réseaux existants. La fonction 
wifi\_\-scan\_\-network effectue un scan pendant 3 secondes sur une interface en mode monitor. La fonction wifi\_\-scan\_\-all\_\-frequencies
effectue le même scan mais sur toutes les fréquences de la bande à 2.4GHz. Le scan se fait en utilisant les fonctions de la 
librairie pcap. Celle-ci n'utilise pas les codes d'erreur mais demande à ce que l'utilisateur fournisse un pointeur sur un tableau de
caractères dans lequel sera écrit une description de l'erreur si besoin. Nous avons donc du reprendre ce principe pour les fonctions
wifi\_\-scan\_\-network et wifi\_\-scan\_\-all\_\-frequencies.

\section{Adressage IP}

Les fonctions relatives à l'adressage IP sont définies dans le fichier ip.c. La fonction wifi\_\-start\_\-avahi permet de lancer un deamon
avahi sur une interface. Il est possible de spécifier en paramètre quelle adresse IP avahi doit tenter de réserver en priorité. La 
fonction wifi\_\-stop\_\-avahi permet au contraire d'arrêter le deamon avahi. La fonction wifi\_\-get\_\-ip\_\-address permet de récupérer l'adresse
IP associée à une interface. Elle utilise l'API libnl-route pour envoyer des messages netlink et en analyser le résultat. Elle  le donne
sous forme d'un tableau de 4 octets. La fonction print\_\-ip\_\-address permet d'afficher sous forme standart\footnote{quatre 
nombres entre 0 et 255 séparés par un point} une adresse donnée par 4 octets en prenant en paramètre un pointeur sur le premier de ces
octets.

\section{Routage}

Nous avons créé une structure permettant de représenter une entrée dans la table des chemins. Celle-ci est définie dans le fichier
mpath.h. Chaque entrée contient donc une destination et un prochain saut, sous la forme d'une adresse mac au format 6 octets, et 
une distance qui est un int. Pour représenter la table des chemins, il est nécessaire de pouvoir organiser ces entrées sous forme 
d'une liste, c'est pourquoi nous avons ajouté à la structure une variable list\_\-head. Le fichier mpath.c contient donc les fonctions
permettant de gérer cette structure.

Le fichier route.c contient les fonctions permettant de récupérer et modifier la table des chemins. La fonction wifi\_\-add\_\-mesh\_\-path
permet d'ajouter une entrée dans la table des chemins, wifi\_\-del\_\-mesh\_\-path d'en supprimer une, et wifi\_\-get\_\-mesh\_\-path permet de 
récupérer la liste de toutes les entrées, et donc d'avoir en même temps l'adresse MAC de tous les noeuds constituant le réseau.
Ces oppérations utilisent l'API libnl-genl, c'est pourquoi il est indispensable de donner à ces fonctions une référence sur une
structure wifi\_\-nlstate initialisée, afin de pouvoir utiliser un socket netlink.